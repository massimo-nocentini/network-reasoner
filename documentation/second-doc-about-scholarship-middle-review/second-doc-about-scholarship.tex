\documentclass[twoside,openright,titlepage,fleqn,
headinclude,11pt,a4paper,BCOR5mm,footinclude ]{scrbook}
%--------------------------------------------------------------
        \newcommand{\myTitle}{Analisi di reti metaboliche basata su
          propriet\`a di connessione\xspace}
% use the right myDegree option
\newcommand{\myDegree}{Corso di Laurea in Informatica\xspace}
%\newcommand{\myDegree}{
	%Corso di Laurea Specialistica in Scienze e Tecnologie 
	%dell'Informazione\xspace}
\newcommand{\myName}{Massimo Nocentini\xspace}
\newcommand{\myProf}{Pierluigi Crescenzi\xspace}
\newcommand{\myOtherProf}{Nome Cognome\xspace}
\newcommand{\mySupervisor}{Nome Cognome\xspace}
\newcommand{\myFaculty}{
	Facolt\`a di Scienze Matematiche, Fisiche e Naturali\xspace}
\newcommand{\myDepartment}{
	Dipartimento di Sistemi e Informatica\xspace}
\newcommand{\myUni}{\protect{
	Universit\`a degli Studi di Firenze}\xspace}
\newcommand{\myLocation}{Firenze\xspace}
\newcommand{\myTime}{Anno Accademico 2010-2011\xspace}
\newcommand{\myVersion}{Version 0.1\xspace}
%--------------------------------------------------------------
\usepackage[latin1]{inputenc} 
\usepackage[T1]{fontenc} 
\usepackage[square,numbers]{natbib} 
\usepackage[fleqn]{amsmath}  
\usepackage[english]{babel}
%--------------------------------------------------------------
\usepackage{dia-classicthesis-ldpkg} 
%--------------------------------------------------------------
% Options for classicthesis.sty:
% tocaligned eulerchapternumbers drafting linedheaders 
% listsseparated subfig nochapters beramono eulermath parts 
% minionpro pdfspacing
\usepackage[eulerchapternumbers,subfig,beramono,eulermath,
	parts]{classicthesis}
%--------------------------------------------------------------
\newlength{\abcd} % for ab..z string length calculation
% how all the floats will be aligned
\newcommand{\myfloatalign}{\centering} 
\setlength{\extrarowheight}{3pt} % increase table row height
\captionsetup{format=hang,font=small}
%--------------------------------------------------------------
% Layout setting
%--------------------------------------------------------------
\usepackage{geometry}
\geometry{
	a4paper,
	ignoremp,
	bindingoffset = 1cm, 
	textwidth     = 13.5cm,
	textheight    = 21.5cm,
	lmargin       = 3.5cm, % left margin
	tmargin       = 4cm    % top margin 
}
%--------------------------------------------------------------
\usepackage{listings}
\usepackage{hyperref}
% My Theorem
\newtheorem{oss}{Observation}[section]
\newtheorem{exercise}{Exercise}[section]
\newtheorem{thm}{Theorem}[section]
\newtheorem{cor}[thm]{Corollary}

\newtheorem{lem}[thm]{Lemma}

\newcommand{\vect}[1]{\boldsymbol{#1}}

% questo comando e' relativo alle correzioni che puo
% apportare il prof se lo desidera.
\newcommand{\prof}[1]{\boldsymbol{#1}}

% instead of boldsymbol I can use the arrow above the letter with
%\newcommand{\vect}[1]{\vec{#1}}

% page settings
% \pagestyle{headings}
%--------------------------------------------------------------
\begin{document}
\frenchspacing
\raggedbottom
\pagenumbering{roman}
\pagestyle{plain}
%--------------------------------------------------------------
% Frontmatter
%--------------------------------------------------------------
%\include{titlePage}
\pagestyle{scrheadings}
%--------------------------------------------------------------
% Mainmatter
%--------------------------------------------------------------
\pagenumbering{arabic}

% settings for the lstlisting environment
\lstset{
	language = java
	, numbers = left 
	, basicstyle=\sffamily%\footnotesize
	%, frame=single
	, tabsize=2
	, captionpos=b
	, breaklines=true
	, showspaces=false
	, showstringspaces=false
}
\chapter*{Report of activity: ``Architetture e metodi per la 
cooperazione applicativa'' }
Florence, March 3, 2014
\\\\
This middle review document describes the state of the project under
the scholarship ``Architetture e metodi per la cooperazione
applicativa'', renewed in date $25/12/13$.  The work is supervised by
Prof. Enrico Vicario, the study of fluid dynamics equations and
optimization is due to Fabio Tarani and the implementation of those
concepts is due to Massimo Nocentini.
\\\\
The work in this first three months since renewal covers the following
branches: we have $(i)$ continued the collaboration with Terranova
Software, $(ii)$ enhanced the fluid dynamic engine to support water
contexts, $(iii)$ started to work on a scheduler for the engine. In
the following we describe each of these activities in greater details:
\begin{description}
\item[tuning the engine] the joint work with Terranova Software allows
  us to carefully test and tune the fluid dynamic engine, paying
  special attention to bug fixing and optimization. A man issue has
  been fixed, more specifically the engine is now able to handle the
  computation for the turbulent behavior when Reynolds numbers belongs
  to the interval $[2000, 4000]$. This fix is required for a correct
  computation of gas velocity in pipes.
  
\item[water context enhancement] in order to support our work, we
  enhance the engine toward handling the water context too,
  specializing some behaviors extending the original code base. This
  required not much effort since the pretty code structure (only a new
  class has been created) and the solid theoretical foundation
  proposed by Fabio, which simplify a lot this enhancement.  The main
  new concept introduced in this activity is that of ``tank'', a node
  with the special capability to change its water level during time,
  giving to the entire system a dynamic aspect which isn't present in
  gas networks.
  \newpage
\item[scheduling the engine] in order to ``program'' fixed network
  events (such as do load variations for a node, switch off a
  pipe...), we start to design a scheduler for the water context which
  allow to run the fluid dynamic engine several times driven by the
  set of scheduled events. We're working on a first stage prototype,
  implementing it in \emph{Smalltalk}, while a parallel study of
  existing literature is addressed by Fabio.
  
\end{description}

\begin{tabular}{ l c l }
  written by Massimo Nocentini,  &  & approved by Enrico Vicario,  \\
  Dr. in Computer Science & & Full Professor \\
  \url{massimo.nocentini@gmail.com} &  & \url{enrico.vicario@unifi.it} \\
\end{tabular}

\end{document}
