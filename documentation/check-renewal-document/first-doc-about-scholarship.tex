\documentclass[twoside,openright,titlepage,fleqn,
headinclude,11pt,a4paper,BCOR5mm,footinclude ]{scrbook}
%--------------------------------------------------------------
        \newcommand{\myTitle}{Analisi di reti metaboliche basata su
          propriet\`a di connessione\xspace}
% use the right myDegree option
\newcommand{\myDegree}{Corso di Laurea in Informatica\xspace}
%\newcommand{\myDegree}{
	%Corso di Laurea Specialistica in Scienze e Tecnologie 
	%dell'Informazione\xspace}
\newcommand{\myName}{Massimo Nocentini\xspace}
\newcommand{\myProf}{Pierluigi Crescenzi\xspace}
\newcommand{\myOtherProf}{Nome Cognome\xspace}
\newcommand{\mySupervisor}{Nome Cognome\xspace}
\newcommand{\myFaculty}{
	Facolt\`a di Scienze Matematiche, Fisiche e Naturali\xspace}
\newcommand{\myDepartment}{
	Dipartimento di Sistemi e Informatica\xspace}
\newcommand{\myUni}{\protect{
	Universit\`a degli Studi di Firenze}\xspace}
\newcommand{\myLocation}{Firenze\xspace}
\newcommand{\myTime}{Anno Accademico 2010-2011\xspace}
\newcommand{\myVersion}{Version 0.1\xspace}
%--------------------------------------------------------------
\usepackage[latin1]{inputenc} 
\usepackage[T1]{fontenc} 
\usepackage[square,numbers]{natbib} 
\usepackage[fleqn]{amsmath}  
\usepackage[italian]{babel}
%--------------------------------------------------------------
\usepackage{dia-classicthesis-ldpkg} 
%--------------------------------------------------------------
% Options for classicthesis.sty:
% tocaligned eulerchapternumbers drafting linedheaders 
% listsseparated subfig nochapters beramono eulermath parts 
% minionpro pdfspacing
\usepackage[eulerchapternumbers,subfig,beramono,eulermath,
	parts]{classicthesis}
%--------------------------------------------------------------
\newlength{\abcd} % for ab..z string length calculation
% how all the floats will be aligned
\newcommand{\myfloatalign}{\centering} 
\setlength{\extrarowheight}{3pt} % increase table row height
\captionsetup{format=hang,font=small}
%--------------------------------------------------------------
% Layout setting
%--------------------------------------------------------------
\usepackage{geometry}
\geometry{
	a4paper,
	ignoremp,
	bindingoffset = 1cm, 
	textwidth     = 13.5cm,
	textheight    = 21.5cm,
	lmargin       = 3.5cm, % left margin
	tmargin       = 4cm    % top margin 
}
%--------------------------------------------------------------
\usepackage{listings}
\usepackage{hyperref}
% My Theorem
\newtheorem{oss}{Observation}[section]
\newtheorem{exercise}{Exercise}[section]
\newtheorem{thm}{Theorem}[section]
\newtheorem{cor}[thm]{Corollary}

\newtheorem{lem}[thm]{Lemma}

\newcommand{\vect}[1]{\boldsymbol{#1}}

% questo comando e' relativo alle correzioni che puo
% apportare il prof se lo desidera.
\newcommand{\prof}[1]{\boldsymbol{#1}}

% instead of boldsymbol I can use the arrow above the letter with
%\newcommand{\vect}[1]{\vec{#1}}

% page settings
% \pagestyle{headings}
%--------------------------------------------------------------
\begin{document}
\frenchspacing
\raggedbottom
\pagenumbering{roman}
\pagestyle{plain}
%--------------------------------------------------------------
% Frontmatter
%--------------------------------------------------------------
%\include{titlePage}
\pagestyle{scrheadings}
%--------------------------------------------------------------
% Mainmatter
%--------------------------------------------------------------
\pagenumbering{arabic}

% settings for the lstlisting environment
\lstset{
	language = java
	, numbers = left 
	, basicstyle=\sffamily%\footnotesize
	%, frame=single
	, tabsize=2
	, captionpos=b
	, breaklines=true
	, showspaces=false
	, showstringspaces=false
}

\chapter*{First doc about scholarship: ``a'' }
\begin{description}
\item[Fellow] Massimo Nocentini, \url{massimo.nocentini@gmail.com}
\item[Professor] Enrico Vicario, \url{enrico.vicario@unifi.it}
\end{description}

In this document we describe the state of the project under ``put here the name
of the founding''. The work is supervised by Prof. Enrico Vicario, the study of
fluid dynamics equations and optimizations is due to Eng. Fabio Tarani and
implemented by Doct. Massimo Nocentini using the $C\#$ programming language.
\\\\
Problem: write a library that, consuming a fluid dynamics network specification 
containing topological connections and physical data such as loads and
setup pressures on nodes, produces for each node the relative pressure and for
each edge the relative flow.
\\\\
In the beginning stage of the project there was a phase of study, performed in
two distinct branches: $(i)$ study of existing product ``GeoNet'' supplied by
Terranova Software, $(ii)$ analysis of suitable object-oriented implementation
for the network concept. These branches joined together with the study of equations
about fluid dynamics context. Since that point the implementation phase has 
dominated the others and is currently active despite it is quite stable.

We describe the implementation and proceed step by step, from the architecture
toward input/output, observing the main points of interest:
\begin{description}
\item[architecture] the problem of describing the input network is solved using
a ``language'' based on the idea of inductive set construction. We provide an
example regarding the definition of nodes set:
\begin{itemize}
	\item a topological node $t$ with identifier $id$ and height $h$ is a node;
	\item if $n$ is a node then a node $m$ decorating $n$ with a gadget $g$ 
		(which in turn is defined by induction), say with setup pressure $p$, 
		 is a node. 
\end{itemize}

Using this approach to define \emph{every} object of the network under study
allow us to tackle the problem from a functional point of view. In fact, given a
network specified using the above schema, it is possible to create dedicated
sub domains, each relative to an aspect of interest, extracting the
necessary informations only. For example it
is possible to generate a domain for the visual representation of the network or
to generate a domain for the Newton-Raphson algorithm in order to compute
pressures and flows and so on.

\item[Visitor pattern]
the process of sub domain creation (aka a function over a recursively defined
type) is the counterpart of a proof by structural
induction done in mathematics. This creation process is implemented using the
\emph{Visitor} design pattern, which has the following pros:
\begin{itemize}
	\item allow the implementation of a domain creation to be written in a
		single class;
	\item decouple the type's variants from the operations which can be 
		defined over them, keeping the variants' code clean;
\end{itemize}
We've experienced a cons of this approach regarding the possibility of adding a
new variant, since we have to resort to a downcast to type check our implementation.

\item[input specification] it is possible to specify the input instance using
two main formalisms:
\begin{itemize}
	\item \emph{RDF triples}, this formalism is based on the W3C
		recommendation for data description in the semantic web. It is a
		verbose input specification but allow a user friendly way to
		describe the network;
	\item \emph{GeoNet format}, this formalism is used by the existing
		engine and is a ``quick and dirty'' way to define the network,
		without the overhead of RDF triples notation but less
		readable.
\end{itemize}

\item[Newton-Raphson algorithm] we implemented this algorithm to compute
pressures on nodes and flows on edges by consecutive iterations. 
Actually we coded a variant which use
$(i)$ a theoretical idea about smart Jacobi an matrix construction reached by Fabio
Tarani and $(ii)$ a relaxing factor in order to cut down the number of
iterations due to oscillation of some variables.

\item[pure $C\#$ library] the entire code base has been implemented using
\emph{only} $C\#$ code, paying attention to focus on the ``message passing''
paradigm which is the base for true object orientation. We promote delegation
over inheritance and don't use ``exotic'' features of $C\#$
such as coroutines, threads and external process invocation in favor of a 
minimal core. We divided the solution in many little projects in order to be 
modular and, possibly, allow reuse in other projects. It is important to note that
Terranova Software's projects implementation is completely written in $C\#$ and
currently the GeoNet engine, written in $C++$ instead, is invoked as an external
process: using our implementation it is possible to have a complete solution running 
on .NET platform without resorting to foreign calls.

\item[simulations] we've exercised the library against some test networks, some
of them are simple and are to be considered as ``toy'' networks in order to
check the consistency of the computation. In the last month Fabio Tarani found a
network which abstract a real context of a Polish town: running the analysis for
that network we obtain satisfactory results both for execution time both for the
correctness of computed pressures and flows. Comparing with GeoNet our
implementation computes smaller values for pressures.
\end{description}


\end{document}
